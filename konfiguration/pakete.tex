\documentclass[chapterprefix=false, 12pt, a4paper, oneside, parskip=half, listof=totoc, bibliography=totoc, numbers=noendperiod, headinclude, twoside]{scrreprt}

%Zeilenabstand einstellen
\usepackage{setspace}

%leertext einfügen
\usepackage{lipsum}

% Für die Erstellung des Hintergrundbilds / shipout befehl
\usepackage{eso-pic}

%Anpassung der Seitenränder (nach Theisen)
\usepackage[top=40mm,bottom=20mm,left=40mm,right=20mm,headsep = \baselineskip, footskip = \dimexpr2\baselineskip+3mm\relax,]{geometry}

% Representation of Directory Structures
\usepackage{dirtree}

%Unterstützung von Umlauten und anderen Sonderzeichen (UTF-8)
\usepackage{lmodern}
\usepackage[utf8]{inputenc}
\usepackage[T1]{fontenc}
\setlength{\emergencystretch}{1em}

% Schriftart Garamond (nach Theisen)
%\usepackage{ebgaramond}

% Schriftart Palatino (nach Theisen)
\usepackage{palatino}

\usepackage{csquotes}
%\usepackage[square,sort,comma,numbers]{natbib}
\usepackage[
backend=biber,
style=alphabetic,
sorting=nyt
]{biblatex}


%\bibliographystyle{agsm}

\usepackage{tabularx} % in the preamble

%Erlaubt unter anderem Umbrüche captions
\usepackage{caption}

%Stichwortverzeichnis
\usepackage{imakeidx}

%Kompakte Listen
\usepackage{paralist}

%Zitate besser formatieren und darstellen
\usepackage{epigraph}

%Ermöglicht Verknüpfungen innerhalb des Dokumentes (e.g. for PDF), Links werden durch "hidelink" nicht explizit hervorgehoben
\usepackage[hidelinks,german]{hyperref}

\usepackage{pdfpages}

%Glossar, Stichwortverzeichnis (Akronyme werden als eigene Liste aufgeführt)
\usepackage[toc, acronym, shortcuts]{glossaries}

%Anpassung von Kopf- und Fußzeile
%beeinflusst die erste Seite des Kapitels
\usepackage[automark, headsepline, autooneside=false]{scrlayer-scrpage}

%Tweaks für scrbook
\usepackage{scrhack}
%\usepackage{fancyhdr}

%\input{resources/styles/header_footer}
\setlength{\parindent}{0em}

%\renewcommand\chapter{\thispagestyle{plain}}

%Auskommentieren für die Verkleinerung des vertikalen Abstandes eines neuen Kapitels
%\renewcommand*{\chapterheadstartvskip}{\vspace*{.25\baselineskip}}
%\renewcommand*{\chapterheadendvskip}{\vspace*{.25\baselineskip}}


%Verbesserte Darstellung der Buchstaben zueinander
\usepackage[stretch=10]{microtype}

%Deutsche Bezeichnungen für angezeigte Namen (z.B. Inhaltsverzeichnis etc.)
\usepackage[ngerman]{babel}

%Einfachere Zitate
\usepackage{epigraph}

%Unterstützung der H Positionierung (keine automatische Verschiebung eingefügter Elemente)
\usepackage{float}

%Erlaubt Umbrüche innerhalb von Tabellen
\usepackage{tabularx}

%Erlaubt Seitenumbrüche innerhalb von Tabellen
\usepackage{longtable}

%Erlaubt die Darstellung von Sourcecode mit Highlighting
\usepackage{listings}

%Vektorgrafiken
%\usepackage{tikz}

%Grafiken (wie jpg, png, etc.)
\usepackage{graphicx}

%Grafiken von Text umlaufen lassen
\usepackage{wrapfig}

\usepackage[dvipsnames]{xcolor}

% Checkmarks
\usepackage{pifont}

%Pseudocode einfügen
\usepackage{algorithm}
\usepackage{algpseudocode}

%Für lange Tabellen
\usepackage{longtable}
\usepackage{longfigure}
\usepackage{capt-of}