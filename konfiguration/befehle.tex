%\newcommand*{\footcite}[1]{\footnote{\citep{#1}}}

% Hintergrundbild einfügen
\newcommand\BackgroundPic{%
\put(0,0){%
\parbox[b][\paperheight]{\paperwidth}{%
\vfill
\centering
\includegraphics[width=\paperwidth,height=\paperheight,%
keepaspectratio]{images/deckblatt.png}%
\vfill
}}}

% Neuer Befehl um Subsubsubkapitel schreiben zu können
\newcommand{\subsubsubsection}[1]{\paragraph{#1}\mbox{}\\}

% Schriftgröße der Fußnote ändern
\renewcommand{\footnotesize}{\normalsize}

%Hilfsbefehl für Glossar UND Abkürzung
\newcommand{\newdefinedabbreviation}[4]{
    \newglossaryentry{#1}{
        text={#2},
        long={#3},
        name={\glsentrylong{#1} (\glsentrytext{#1})},
        first={\glsentryname{#1}},
        firstplural={\glsentrylong{#1}\glspluralsuffix (\glsentryname{#1}\glspluralsuffix )},
        description={#4}
    }
}

\newcommand{\cmark}{\ding{51}}%
\newcommand{\xmark}{\ding{55}}%

\newcommand*{\newdualentry}[5][]{%
  \newglossaryentry{main-#2}{name={#4},%
  text={#3\glsadd{#2}},%
  description={{#5}},%
  #1
  }%
  \newglossaryentry{#2}{
  type=\acronymtype,
  first={#4 (#3)},
  name={#3\glsadd{main-#2}},
  description={\glslink{main-#2}{#4}}
  }%
}

% Erzeugen der Deckblätter
\newcommand{\makeTitles}{

% Entfernen der Seitenzahlen und BackroundPic als Hintergrund nutzen
\pagenumbering{gobble}
\AddToShipoutPicture{\BackgroundPic}

% Variablen für das Deckblatt 1
\gradeType{Bachelor of Science (B.Sc.)}
\germanTitle{Deutscher Titel}
\englishTitle{Englischer Titel}
\authorFirstname{Vorname Autor1}
\authorLastname{Nachname Autor1}
\authorBirthplace{Geburtsort Autor1}
\discipline{Fachbereich}
\courseOfStudies{Studiengang}
\matrikelnr{Matrikelnummer}
\submitDate{{\today}}
\firstExaminer{Prof. Dr. Erster Prüfer}
\secondExaminer{Prof. Dr. Zweiter Prüfer}

% Titelblatt 1 erzeugen
\maketitle
\let\cleardoublepage\clearpage
% Eidesstattliche Erklärung 1 erzeugen
\makeeidesstatt
\let\cleardoublepage\clearpage

% Variablen für das Deckblatt 2
\gradeType{Bachelor of Science (B.Sc.)}
\germanTitle{Deutscher Titel}
\englishTitle{Englischer Titel}
\authorFirstname{Vorname Autor2}
\authorLastname{Nachname Autor2}
\authorBirthplace{Geburtsort Autor2}
\discipline{Fachbereich}
\courseOfStudies{Studiengang}
\matrikelnr{Matrikelnummer}
\submitDate{{\today}}
\firstExaminer{Prof. Dr. Erster Prüfer}
\secondExaminer{Prof. Dr. Zweiter Prüfer}

% Titelblatt 2 erzeugen
\maketitle

% Eidesstattliche Erklärung 2 erzeugen
\makeeidesstatt

\ClearShipoutPicture
}